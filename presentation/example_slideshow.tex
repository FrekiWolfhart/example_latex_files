\documentclass{beamer} % Used to create a slideshow
\usepackage{graphicx}

% Classical title/author/date thing
\title{Test of Beamer}
\author{No one}
\date{}

\usetheme{Frankfurt} % Beamer themes for the slideshow

\begin{document}
\maketitle

\section{Introduction}
% In a slide show, sections create bullet points wich allows you to move between them easily
\begin{frame} % Create a new frame, i.e. a new slideshow
\frametitle{Test} % Puts the title of the frame

\begin{itemize}
	\item cat \pause % Pause is used to make elements appear one after the other
	\item wolf \pause % With pause, you need to change slides to make the rest appear
	\item lizard \pause
	\item math % There is no point in making a pause when there is nothing else to print in the slide
\end{itemize}
\end{frame}

\section{Felids}
\begin{frame}
\frametitle{cat}
Cats are felids, which is a categories of animals part of the felidae family. % Text is like normale LaTeX
\end{frame}

\section{Image}
\begin{frame}
\includegraphics{images/image.png} %Works just like regular LaTeX
\end{frame}

\section{Columns}
\begin{frame}
\frametitle{Seaprate texts}
\begin{columns}
\column{.5\textwidth}
This is a column, the left one, having exactly the first half of the slide for itself.
\column{.5\textwidth}
This is an other column, the right one, having the other half of the slide.
\end{columns}
\end{frame}

\end{document}
